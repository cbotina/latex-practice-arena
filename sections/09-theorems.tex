\section{Theorems and Definitions}

\subsection{Definitions}

\begin{definition}[Convergence]
A sequence $\{x_n\}$ converges to a limit $L$ if for every $\epsilon > 0$, there exists a natural number $N$ such that for all $n > N$, we have $|x_n - L| < \epsilon$.
\end{definition}

\subsection{Theorems}

\begin{theorem}[Fundamental Theorem of Calculus]
If $f$ is continuous on $[a,b]$ and $F$ is an antiderivative of $f$ on $[a,b]$, then
\begin{equation}
    \int_a^b f(x) \, dx = F(b) - F(a)
\end{equation}
\end{theorem}

\begin{lemma}
For any positive real numbers $a$ and $b$, we have:
\begin{equation}
    \sqrt{ab} \leq \frac{a + b}{2}
\end{equation}
with equality if and only if $a = b$.
\end{lemma}

